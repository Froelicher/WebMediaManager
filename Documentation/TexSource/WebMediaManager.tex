% !TEX TS-program = pdflatex
% !TEX encoding = UTF-8 Unicode

% This is a simple template for a LaTeX document using the "article" class.
% See "book", "report", "letter" for other types of document.

\documentclass[11pt]{report} % use larger type; default would be 10pt

\usepackage[utf8]{inputenc} % set input encoding (not needed with XeLaTeX)

%%% Examples of Article customizations
% These packages are optional, depending whether you want the features they provide.
% See the LaTeX Companion or other references for full information.

%%% PAGE DIMENSIONS
\usepackage{geometry} % to change the page dimensions
\geometry{a4paper} % or letterpaper (US) or a5paper or....
% \geometry{margin=2in} % for example, change the margins to 2 inches all round
% \geometry{landscape} % set up the page for landscape
%   read geometry.pdf for detailed page layout information

\usepackage{graphicx} % support the \includegraphics command and options

% \usepackage[parfill]{parskip} % Activate to begin paragraphs with an empty line rather than an indent

%%% PACKAGES
\usepackage{booktabs} % for much better looking tables
\usepackage{array} % for better arrays (eg matrices) in maths
\usepackage{paralist} % very flexible & customisable lists (eg. enumerate/itemize, etc.)
\usepackage{verbatim} % adds environment for commenting out blocks of text & for better verbatim
\usepackage{subfig} % make it possible to include more than one captioned figure/table in a single float
% These packages are all incorporated in the memoir class to one degree or another...

%%% HEADERS & FOOTERS
\usepackage{fancyhdr} % This should be set AFTER setting up the page geometry
\pagestyle{fancy} % options: empty , plain , fancy
\renewcommand{\headrulewidth}{0pt} % customise the layout...
\lhead{}\chead{}\rhead{}
\lfoot{}\cfoot{\thepage}\rfoot{}

%%% SECTION TITLE APPEARANCE
\usepackage{sectsty}
\allsectionsfont{\sffamily\mdseries\upshape} % (See the fntguide.pdf for font help)
% (This matches ConTeXt defaults)

%%% ToC (table of contents) APPEARANCE
\usepackage[nottoc,notlof,notlot]{tocbibind} % Put the bibliography in the ToC
\usepackage[titles,subfigure]{tocloft} % Alter the style of the Table of Contents
\renewcommand{\cftsecfont}{\rmfamily\mdseries\upshape}
\renewcommand{\cftsecpagefont}{\rmfamily\mdseries\upshape} % No bold!

%%% END Article customizations

%%% The "real" document content comes below...

\title{Brief Article}
\author{The Author}
%\date{} % Activate to display a given date or no date (if empty),
         % otherwise the current date is printed 

\begin{document}
\maketitle

\chapter{Résumés}

\chapter{Introduction}

\chapter{Cahier des charges}

\chapter{Étude d'opportunité}
\newpage

	\section{Introduction au projet}
	Le but de ce projet est de réaliser une application permettant l'utilisation des différents services proposé par les sites de vidéos et de diffusion de flux vidéo en direct.
		\subsection{Médias vidéo web}
		 Aujourd'hui, les sites comme Youtube, Dailymotion, Twitch qui propose du contenu vidéos sont de plus en plus visités. Ses sites, ont tous un point commun, d'autres personnes mettent des vidéos en ligne pour divertir les spectateurs. Le phénomène a prit une tel ampleur que certain "créateur de contenu audiovisuel" sont même 	payé par ses sites par rapport à leur popularité. De plus en plus de personnes, surtout les jeunes, passent leurs temps devant des vidéos sur le web que sur la télévision.
			\subsubsection{Direct}
			Les vidéos en direct sont de plus en plus présent sur le web. En effet, grâce en grande partie aux jeux vidéos, le phénomène des personnes créant du contenu audiovisuel c'est également répandu sur du contenu en direct. Des sites comme Twitch ou Dailymotion propose à ses utilisateurs de diffuser du flux vidéo en direct. La plus part du temps c'est pour les jeux vidéos, les diffuseurs jouent sur leurs ordinateurs / consoles puit retransmet l'image sur le site. Ainsi, des personnes du monde entier ont la possibilité de regarder la partie de jeu d'une personne. Les spectateurs ont même la possibilité de discuter avec les autres spectateurs et même des fois avec le diffuseur de contenu.
Les diffuseurs sont payés grâce aux dons de leurs communauté, certain font ça à plein temps et donc sont contraint à entretenir une grande communauté.
			\subsubsection{Différé}
			Les vidéos "différé" sont depuis un petit moment déjà présente sur le web. En effet, le phénomène des vidéos sur le web n'est pas tout nouveau mais, ce n'est que depuis peu que le phénomène à prit une grande ampleur. Si un créateur de contenu à atteint une certaine popularité, il peut enfin prétendre à faire de l'argent avec ses vidéos. En effet, le premier site de vidéos du monde Youtube rémunère les créateurs de contenu.

		\subsection{Pourquoi avoir choisi ce sujet ?}
		J'ai choisi ce sujet car je porte un réel interêt au monde audiovisuel sur le web. En effet, je fais parti des jeunes qui a délaissé la télévision pour les vidéos sur internet. Il y a plusieurs site de vidéos et des fois il est difficile de s'y retrouver, l'application que j'ai pensé a pour but de facilité la vie des personnes comme moi qui utilise ses sites régulièrement. Le fait que le monde de l'audiovisuel est en plein boum sur le web me stimule encore plus à l'idée de créer une application dans ce domaine.
	\section{Analyse de l'exsistant}
		\subsection{Existant}
		\subsection{Critique de l'existant}	

\chapter{Analyse fonctionnelle}
	\section{Interface homme-machine}

\chapter{Analyse organique}

\chapter{Tests}
	\section{Tests fonctionnel}
	\section{Tests unitaire}
\chapter{Conclusions}



Your text goes here.

\subsection{A subsection}

More text.

\end{document}
